\begin{landscape}
\begin{figure}[p]
\centering
\vspace*{-0.8cm} % Adjust for header space
\begin{tikzpicture}[
    scale=1.0,
    transform shape,
    block/.style={rectangle, draw=black, thick, fill=blue!20, text width=2.8cm, 
                  text centered, minimum height=1.4cm, font=\scriptsize, inner sep=4pt},
    smallblock/.style={rectangle, draw=black, thick, fill=green!20, text width=2.4cm, 
                       text centered, minimum height=1.2cm, font=\scriptsize, inner sep=4pt},
    memory/.style={rectangle, draw=black, thick, fill=orange!20, text width=2.4cm, 
                   text centered, minimum height=1.2cm, font=\scriptsize, inner sep=4pt},
    arrow/.style={->, thick, blue!70!black},
    bidiarrow/.style={<->, thick, red!70!black}
]

% Central Memory Traversal Unit - center of diagram
\node[block] (mtu) at (0,0) {Memory\\Traversal\\Unit (MTU)};

% Execute Module - to the right with moderate spacing
\node[block] (execute) at (4.0,0) {Execute\\Module};

% Specialized Operation Blocks - stacked vertically on far right with more spacing
\node[smallblock] (cell) at (8.5,1.8) {Cell Block\\(Op 3)};
\node[smallblock] (inc) at (8.5,0.6) {Increment\\(Op 4)};
\node[smallblock] (equal) at (8.5,-0.6) {Equal\\(Op 5)};
\node[smallblock] (edit) at (8.5,-1.8) {Edit\\(Op 10)};

% Memory Components - stacked vertically on left with much more spacing
\node[memory] (memunit) at (-5,1.8) {Memory\\Unit (MU)};
\node[memory] (memmux) at (-5,0) {Memory\\Multiplexer};
\node[memory] (gc) at (-5,-1.8) {Garbage\\Collector};

% Control Components - below center with more space
\node[smallblock] (ctrlmux) at (2,-3) {Control\\Multiplexer};

% Program and Back Pointers - bottom left with more space
\node[smallblock] (pointers) at (-1,-3) {P \& B\\Pointers};

% Data flow arrows - bidirectional with better routing
\draw[bidiarrow] (mtu) -- (execute) node[midway, above, sloped] {\scriptsize control};
\draw[bidiarrow] (mtu) -- (memunit) node[midway, above, sloped] {\scriptsize mem ops};
\draw[bidiarrow] (mtu) -- (memmux) node[midway, above, sloped] {\scriptsize access};
\draw[bidiarrow] (mtu) -- (ctrlmux) node[midway, above, sloped] {\scriptsize state};
\draw[bidiarrow] (mtu) -- (pointers) node[midway, above, sloped, sloped] {\scriptsize traversal};

% Execute module to operation blocks - simple direct lines with better spacing
\draw[arrow] (execute) -- (cell) node[midway, above, sloped] {\scriptsize dispatch};
\draw[arrow] (execute) -- (inc) node[midway, above, sloped] {\scriptsize dispatch};
\draw[arrow] (execute) -- (equal) node[midway, below, sloped] {\scriptsize dispatch};
\draw[arrow] (execute) -- (edit) node[midway, below, sloped] {\scriptsize dispatch};

% Garbage collection trigger - routed far above to avoid all overlaps
  \draw[arrow] (mtu.south) -- (gc.north) node[midway, above, sloped] {\scriptsize gc trigger};

% Memory hierarchy connections - vertical on left side
\draw[bidiarrow] (memunit) -- (memmux) node[midway, right] {\scriptsize mux};
\draw[bidiarrow] (memmux) -- (gc) node[midway, right] {\scriptsize gc mem};

% Section labels positioned to avoid overlap with gc trigger line
\node[above=1.2cm of mtu, font=\small\bfseries] {NockPU Core};
\node[above=0.6cm of execute, font=\small\bfseries] {Execution Units};
\node[above=0.6cm of memunit, font=\small\bfseries] {Memory Subsystem};

\end{tikzpicture}
\vspace*{-0.5cm} % Adjust for footer space
\caption{NockPU System Architecture Overview. The Memory Traversal Unit serves as the central controller, coordinating between execution modules and memory management components. Specialized operation blocks handle specific Nock opcodes, while the memory subsystem provides cell-based storage with integrated garbage collection. Bidirectional arrows (red) indicate control and data flow, while unidirectional arrows (blue) show dispatch operations.}
\label{fig:nockpu-architecture}
\end{figure}
\end{landscape}
