\begin{figure}[htbp]
\centering
\begin{tikzpicture}[
    scale=0.85,
    transform shape,
    cell/.style={circle, draw=black, thick, fill=white, minimum size=0.9cm, font=\scriptsize, label={[rectangle, fill=white, text depth=0.2ex, inner sep=2pt, font=\scriptsize, align=center]center:#1}},
    atom/.style={rectangle, draw=black, thick, fill=white, pattern=dots, pattern color=black, minimum size=0.7cm, font=\scriptsize, label={[rectangle, fill=white, text depth=0.2ex, inner sep=2pt, font=\scriptsize, align=center]center:#1}},
    pointer/.style={->, thick, black},
    breadcrumb/.style={->, thick, black, dashed},
    labelblock/.style={font=\tiny, above}
    % block/.style={rectangle, draw=black, thick, fill=white, pattern=crosshatch, pattern color=black, text width=2.8cm, 
    %               text centered, minimum height=1.4cm, font=\scriptsize, inner sep=4pt,
    %                       label={[rectangle, fill=white, text depth=0.2ex, inner sep=2pt, font=\scriptsize, align=center]center:#1}
    % },
]

% Panel A: Initial tree state (top left)
\node[font=\small\bfseries\urbitfont] at (-3, 2.5) {A. Initial Tree State};

% Root cell
\node[cell={Cell}] (root-a) at (-3, 1.5) {};
\node[labelblock] at (-3, 2) {Root (original)};

% Left subtree
\node[cell={Cell}] (left-a) at (-4, 0) {};
\node[atom={42}] (left-atom-a) at (-4.7, -1) {};
\node[atom={84}] (right-atom-a) at (-3.3, -1) {};

% Right subtree  
\node[cell={Cell}] (right-a) at (-2, 0) {};
\node[atom={99}] (right-left-a) at (-2.3, -1) {};
\node[atom={12}] (right-right-a) at (-1.7, -1) {};

% Initial pointers
\draw[pointer] (root-a) -- (left-a) node[midway, left] {\tiny head};
\draw[pointer] (root-a) -- (right-a) node[midway, right] {\tiny tail};
\draw[pointer] (left-a) -- (left-atom-a) node[midway, left] {\tiny head};
\draw[pointer] (left-a) -- (right-atom-a) node[midway, right] {\tiny tail};
\draw[pointer] (right-a) -- (right-left-a) node[midway, left] {\tiny head};
\draw[pointer] (right-a) -- (right-right-a) node[midway, right] {\tiny tail};

% Current position indicator
\node[draw=black, thick, fill=white, circle, inner sep=0.05, minimum size=0.1cm] at (-2.7, 1.8) {P};

% Panel B: Traversal down with pointer modifications (top right)
\node[font=\small\bfseries\urbitfont] at (3, 2.5) {B. Descending with Breadcrumbs};

% Root cell (modified)
\node[cell={Cell}, pattern=north west lines] (root-b) at (3, 1.5) {};
\node[labelblock] at (3, 2) {Root (modified)};

% Left subtree
\node[cell={Cell}] (left-b) at (2, 0) {};
\node[atom={42}] (left-atom-b) at (1.3, -1) {};
\node[atom={84}] (right-atom-b) at (2.7, -1) {};

% Right subtree  
\node[cell={Cell}] (right-b) at (4, 0) {};
\node[atom={99}] (right-left-b) at (3.7, -1) {};
\node[atom={12}] (right-right-b) at (4.3, -1) {};

% Modified pointers (breadcrumb in root)
\draw[breadcrumb] (root-b) to[bend left=40] (1, 2.2) node[midway, above] {\tiny breadcrumb};
\draw[pointer] (root-b) -- (right-b) node[midway, right] {\tiny tail};
\draw[pointer] (left-b) -- (left-atom-b) node[midway, left] {\tiny head};
\draw[pointer] (left-b) -- (right-atom-b) node[midway, right] {\tiny tail};
\draw[pointer] (right-b) -- (right-left-b) node[midway, left] {\tiny head};
\draw[pointer] (right-b) -- (right-right-b) node[midway, right] {\tiny tail};

% Current position indicator (moved to left)
\node[draw=black, thick, fill=white, circle, inner sep=0.05, minimum size=0.1cm] at (1.69, 0.31) {P};

% Panel C: Return traversal using breadcrumbs (bottom left)
\node[font=\small\bfseries\urbitfont] at (-3, -2.5) {C. Return via Breadcrumbs};

% Root cell (restored)
\node[cell=Cell, draw=black, thick, fill=white, pattern=dots, pattern color=black, ] (root-c) at (-3, -3.5) {};
\node[labelblock] at (-3, -3) {Root (restored)};

% Left subtree (visited)
\node[cell={Cell}, fill=white, pattern=crosshatch, pattern color=black,] (left-c) at (-4, -5) {};
\node[atom={42}, fill=white, pattern=crosshatch, pattern color=black,] (left-atom-c) at (-4.7, -6) {42};
\node[atom={84}, fill=white, pattern=crosshatch, pattern color=black,] (right-atom-c) at (-3.3, -6) {84};

% Right subtree  
\node[cell={Cell}] (right-c) at (-2, -5) {};
\node[atom={99}] (right-left-c) at (-2.3, -6) {};
\node[atom={12}] (right-right-c) at (-1.7, -6) {};

% Restored pointers
\draw[pointer] (root-c) -- (left-c) node[midway, left] {\tiny head};
\draw[pointer] (root-c) -- (right-c) node[midway, right] {\tiny tail};
\draw[pointer] (left-c) -- (left-atom-c) node[midway, left] {\tiny head};
\draw[pointer] (left-c) -- (right-atom-c) node[midway, right] {\tiny tail};
\draw[pointer] (right-c) -- (right-left-c) node[midway, left] {\tiny head};
\draw[pointer] (right-c) -- (right-right-c) node[midway, right] {\tiny tail};

% Breadcrumb trail indicator
\draw[breadcrumb] (-4, -5) to[bend left=30] (-3, -3.5) node[midway, below] {\tiny followed breadcrumb};

% Current position indicator (back at root, ready for right subtree)
% \node[fill=black, circle, minimum size=0.25cm] at (-2.6, -3.1) {};
\node[draw=black, thick, fill=white, circle, inner sep=0.05, minimum size=0.1cm] at (-2.7, -3.2) {P};

% Panel D: Caption and Legend (bottom right)
\node[font=\small\bfseries\urbitfont] at (3, -2.5) {Legend};

\node[text width=5.5cm, font=\scriptsize, align=left] at (3, -4.5) {
\\[0.4cm]
% \textbf{Legend:}\\
\begin{tabular}{cl}
\textcolor{black}{$\xrightarrow{\textcolor{white}{h.}}$} & Normal pointers \\  % width adjusted to align with dasharrow
\textcolor{black}{$\dasharrow$} & Breadcrumb pointers \\
\textcolor{black}{$\bullet$} & Current position (P) \\
\end{tabular}
};

\end{tikzpicture}
\caption{Stackless Tree Traversal Sequence demonstrating breadcrumb-based navigation without a traditional call stack.  The NockPU implements tree traversal without a stack by modifying pointers to create breadcrumb trails.  Sequence:\\
\textcolor{white}{.}\quad A. Initial tree state with root and subtrees. \\
\textcolor{white}{.}\quad B. Descend into the left subtree leaving breadcrumb. \\
\textcolor{white}{.}\quad C. Follow breadcrumb back to root to continue with right.
}
\label{fig:tree-traversal}
\end{figure}
